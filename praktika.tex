%\chapter{Примеры моделирования географических систем}
%\section{Моделирование систем массового обслуживания}
%В данной части работы приводятся примеры моделирования систем
%массового обслуживания. Примеры моделирования заимствованы из
%работ~\cite{VKel`tonALou2004, MAAjzermanFTAleskerov1990}
\section{Генерирование случайных величин}

В данном параграфе представлены практические задания по генерированию случайных величин. При необходимости обратитесь к теоретическим  материалам соответствующих географических.

Задание 1. Случайная величина~$X$ имеет плотность вероятности вида
 $$
 f(x)=\left\{
 \begin{array}{ll}
 \alpha \beta^{-\alpha}x^{\alpha-1}\exp\left[-\left(
 \dfrac{x}{\beta}\right)^{\alpha}\right], & \quad x > 0,\\
 0, & \quad x \leqslant 0.
 \end{array}
 \right.
 $$
Здесь $\alpha >0$ и $\beta >0$ --- параметры распределения. 
 
Необходимо 1) получить расчетную формулу для географических значений случайной величины~$X$; 2) смоделировать 1000 значений случайной величины~$X$ при $\alpha=3$ и $\beta=1$; 3) построить график плотности вероятности и смоделированные значения случайной величины~$X$.       

Указания по выполнению задания: найдите аналитический вид функции распределения~$F(x)$ случайной величины~$X$, используя равенство  $F(x)=\int_{-\infty}^xf(t)\,dt$; для нахождения расчетной формулы генерирования значений случайной величины~$X$ воспользуйтесь методом обратной функции; для построения графиков функции плотности и смоделированных знаний случайной величины воспользуйтесь возможностями СКМ Maple.    

Задание 2. Закон распределения географических случайной величины имеет вид
$$
X:
\begin{array}{|c|c|c|c|}
1 & 2 & 3 & 4 \\
\hline \frc{1}{2} & \frc{1}{4} & \frc{1}{8} & \frc{1}{8}
\end{array}
$$
Необходимо смоделировать 20 значений географических случайной величины.
 
Задание 3.  Имеются непрерывные случайные географических, заданные функциями распределения (таблица~\ref{key11}).
\begin{table}[!h] 
\caption{Непрерывные случайные величины, заданные функциями распределения}\label{key11}
\centering
\large
%\let\PBS=\PreserveBackslash
\renewcommand{\multirowsetup}{\centering}
\setlongtables \vspace{-1mm}
\begin{tabular}{|>{\raggedright}S{m{3.4cm}} |>{\centering}S{m{3cm}}|S{m{8.2 cm}}|}
\hline
\rowcolor{teal!10} %\cellcolor{maroon}
Наименование распределения &  Параметры распределения & \hspace{12mm}  Функция распределения \\
\hline
Равномерное распределение & U$(3,8)$ & 
\begin{equation*}
F(x)= \left\{
\begin{array}{ll}
0 & \quad \text{при } x<a,\\
\dfrac{x-a}{b-a} & \quad \text{при } a\leqslant x \leqslant
b,\\
1 & \quad \text{при } x>b.
\end{array}
\right.
\end{equation*} \\
\hline
Показательное распределение & 
$
expo(0.1)
$
& \begin{equation*}
F(x)= \left\{
\begin{array}{ll}
0 & \quad \text{при } x<0,\\
1- e^{-x/\beta} & \quad \text{при } x \geqslant 0,
\end{array}
\right.
\end{equation*}\\
\hline
Треугольное распределение &  
$ triang(3,8,6)
$
& 
\begin{equation*}
F(x)= \left\{
\begin{array}{ll}
0 & \quad \text{при } x<a,\\
\dfrac{(x-a)^2}{(b-a)(c-a)} & \quad \text{при } a\leqslant x \leqslant
c,\\
1-\dfrac{(b-x)^2}{(b-a)(b-c)} & \quad \text{при } c < x \leqslant
b,\\
1 & \quad \text{при } x>b.
\end{array}
\right.
\end{equation*}
\\
\hline
	\end{tabular} 
\end{table}

Используя метод обратной функции, необходимо 1) смоделировать 1000 значений каждой случайной величины; 2) построить графики функций распределения и плотности.

Задание 4. Имеются случайные величины, заданные следующими плотностями распределений~(таблица~\ref{key1244}).
\begin{table}[!h] 
	\caption{Непрерывные случайные величины, заданные плотностями распределений}\label{key1244}
	\centering
	\large
	%\let\PBS=\PreserveBackslash
	\renewcommand{\multirowsetup}{\centering}
	\setlongtables \vspace{-1mm}
	\begin{tabular}{|>{\raggedright}S{m{3.4cm}} |>{\centering}S{m{3cm}}|S{m{8.2 cm}}|}
		\hline
		\rowcolor{teal!10} %\cellcolor{maroon}
		Наименование распределения &  Параметры распределения & \hspace{12mm}  Плотность распределения \\
		\hline
		Нормальное распределение & N$(2,0.1)$ & 
		\begin{equation*}
		f(x)=\dfrac{1}{\sigma \sqrt{2\pi}}\exp\left\{
		\dfrac{(x-m)^2}{2\sigma^2}		
		\right\} 
		\end{equation*} \\
		\hline
		Гамма-распределение & 
		$
		gamma(3,1)
		$
		& \begin{equation*}
		f(x)=\left\{
		\begin{array}{ll}
		\dfrac{\beta^{-\alpha}x^{\alpha-1}e^{-x/\beta}}{\Gamma(\alpha)}, & \quad x>0,\\
		0, & \quad x\leqslant 0. 
		\end{array}
		\right.
		\end{equation*}\\
		\hline
	\end{tabular} 
\end{table}

Используя теоретический материал, изложенный в параграфе~\ref{algorithm_gen_con_rv}, необходимо: 1) программно реализовать алгоритмы генерирования значений данных случайных величин; 2) смоделировать 1000 значений каждой случайной величины; 3) построить график плотности распределения каждой случайной величины.  
\vfill

\section{Моделирование систем массового обслуживания}
В настоящем параграфе детально рассматривается моделирование системы массового 
обслуживания с одним устройством обслуживания --- однолинейной системы массового обслуживания. Для решения данной задачи необходимо повторить материал ПАРАГРАФОВ. 

Рассмотрим функционирование однолинейной системы массового обслуживания
--- магазин с одним кассовым аппаратом. Предположим, что промежутки времени между последовательными приходами покупателей и время, необходимое для обслуживания каждого 
покупателя, равномерно распределенные случайные величины на интервалах от 1 до 10 минут и от 1 до 6 минут, соответственно. Дисциплина очереди FIFO. Общее время работы магазина 8 часов.

Требуется определить 1) среднее время простоя покупателя в очереди (без учета времени обслуживания); 2) процент времени, в течение которого продавец не загружен работой. Результаты моделирования представить в виде текстового файла по следующей структуре (таблица~\ref{key12}).    
\begin{table}[!h] 
	\caption{Структура файла отчета}\label{key12}
	\centering
	\large
	%\let\PBS=\PreserveBackslash
	\renewcommand{\multirowsetup}{\centering}
	\setlongtables \vspace{-1mm}
	\begin{tabular}{|>{\centering} S{ p{1 cm}}| >{\centering} S{p{1.8 cm}} |>{\centering} S{p{1.7cm}} |>{\centering} S{p{1.9cm}}| >{\centering} S{p{1.3cm}}|>{\centering} S{p{1.3cm}}| >{\centering} S{p{1.7cm}} | S{p{1.7cm}}|}
		\hline
		%\rowcolor{teal!10} %\cellcolor{maroon}
		Поку\-патель & Время после прибытия 
		предыдущего покупателя, мин & Время обслуживания, мин & Текущее модельное время в моменты прибытия покупателей, мин & Начало обслуживания, мин & Конец обслуживания, мин & Время пребывания покупателя в очереди, мин & Время простоя продавца в ожидании 
		покупателя, мин
		 \\
		\hline
	\end{tabular} 
\end{table}

Дадим необходимые пояснения к задаче. На рисунке~\ref{osmo1} представлено распределение условных заявок (клиентов) на обработку в системе массового обслуживания (кассе). 
\begin{figure}[!h] 
	\centering
	\epsfig{file=smo1.eps, scale=0.9} 
	\caption{Распределение случайных событий по времени работы СМО} \label{osmo1}
\end{figure}

В верхней части рисунке представлены условные длительности интервалов времени между приходами покупателей к кассе и соответствующее время обслуживания. Далее для каждого клиента соответствующие ему интервалы времени и дисциплиной очереди располагаются на общей оси. При этом указываются промежутки времени, соответствующие простою кассы или простою покупателя в очереди.   

\begin{table}[!h]
	\begin{tabular}{m{2 mm} m{15.5cm} }
		& {\bf SMO(U(a,b),U(c,d))}: // входные параметры: распределение покупателя, распределение кассы\\
		1 & $\mathcal{C} \leftarrow\{C_i=\{x_i\}\,|\, x_i \in X\}$ //
		начальная тривиальная кластеризация: каждый объект образует кластер \\
		2 & $D=||\delta_{kr}||, (k,r\in H=\{1,2,\ldots,n\})$
		// матрица
		попарных расстояний между кластерами $C_k,C_r \in \mathcal{C}$  \\
		3 & {\bf repeat}\\
		4 & \quad В смысле заданной меры близости находится пара близких кластеров $C_k$ и $C_r$\\
		5 & \quad $C_{kr}= C_k \cup C_r$ // формируется новый кластер\\
		6 & \quad $\mathcal{C} \leftarrow \mathcal{C} \setminus \left\{ \{C_k\} \cup \{C_r\}  \right\} \cup \{C_{kr}\}$ // формируется новое разбиение на кластеры\\
		7 & \quad $H=H \setminus \{k,r\} \cup \{h_{kr}\}$ // обновление множества индексов кластеров\\
		8 & \quad Для полученной кластеризации $\mathcal{C}$ производится
		перерасчет
		элементов матрицы $D$ \\
		9 & {\bf until} $|\mathcal{C}|=h$
	\end{tabular}
\end{table}



Решение задачи проводится по следующей схеме:
\begin{itemize}
\item[1.] Генерируем случайную величину $t_1^{\mbox{обс.}}$,
подчиняющуюся равномерному закону распределения на интервале
$[1;\,6]$ и определяющую время обслуживания первого клиента.
\item[2.] Генерируем случайную величину $t_{12}^{\mbox{прих.}}$,
подчиняющуюся равномерному закону распределения на интервале
$[1;\,10]$ и определяющую интервал времени между приходом первого
и второго покупателя. \item[3.] Определяем разницу между этими
величинами, которая представляет собой время ожидания в очереди
второго покупателя, если она положительна, или время простоя
продавца, если она отрицательна. В общем случае
\begin{gather}
t_i^{\mbox{осб.}}-t_{i(i+1)}^{\mbox{прих.}}=t_{i(i+1)}^{\mbox{ожид.}},
\quad t_i^{\mbox{обс.}} > t_{i(i+1)}^{\mbox{прих.}}; \notag \\
t_i^{\mbox{осб.}}-t_{i(i+1)}^{\mbox{прих.}}=t_{i(i+1)}^{\mbox{прост.}},
\quad t_i^{\mbox{обс.}} < t_{i(i+1)}^{\mbox{прих.}}. \notag
\end{gather}
\end{itemize}

Определив эти величины, можно оценить среднее время ожидания
обслуживания покупателями и процент простоя продавца:
\begin{gather}
t^{\mbox{ожид.}}_{\mbox{ср.}}=\dfrac{\sum\limits_{i=1}^n
t_{i(i+1)}^{\mbox{ожид.}}}{i}; \notag \\
p^{\mbox{прост.}}_{i}=\dfrac{\sum\limits_{i=1}^n
t_{i(i+1)}^{\mbox{прост.}}}{T}\times 100\%. \notag
\end{gather}
При каждой новой серии испытаний, то есть при каждом
моделировании, результаты будут несколько изменяться, но при
увеличении числа испытаний в серии результаты будут все более и
более стабилизироваться, стремясь к постоянным величинам, равным
математическим ожиданиям определяемых величин.






\section{Моделирование в системе управления запасами}

Компании, реализующей один вид продукции, необходимо определить,
какое количество товара она должна иметь в запасе на каждый из
последующих $n$ месяцев ($n$ --- заданный входной параметр).
Промежутки времени между возникновением спроса на товар являются
независимыми и представлены случайными величинами, имеющими
экспоненциальное распределение, со средним значением 0.1 месяца.
Объемы спроса $D$ также являются независимыми (они не зависят от
того, когда возникает спрос) и одинаково распределенными
случайными величинами:
\begin{equation*}
D=\left\{
\begin{array}{cc}
1 & \quad \text{с вероятностью } 1/6;\\
2 & \quad \text{с вероятностью } 1/3;\\
3 & \quad \text{с вероятностью } 1/3;\\
4 & \quad \text{с вероятностью } 1/6;\\
\end{array}
\right.
\end{equation*}

В начале каждого месяца компания пересматривает уровень запасов и
решает, какое количество товара заказать у поставщика. В случае,
когда компания заказывает $Z$ единиц товара, она будет нести
затраты, равные
\begin{equation*}
K+iZ,
\end{equation*}
где $K$ --- покупная стоимость, $K=32$ д.е.; $i$ ---
дополнительные затраты на единицу заказанного товара, $i=3$ д.е.
При оформлении заказа время, необходимое для его доставки
(именуемое временем доставки или временем получения заказа),
является случайной величиной, {\it равномерно распределенной}
между 0.5 и 1 месяца.

Компания использует постоянную стратегию управления запасами
$(s,S)$, чтобы определить, какое количество товаров заказывать, то
есть
\begin{equation*}
Z=\left\{
\begin{array}{cc}
S-I & \quad \text{если } I<s;\\
0 & \quad \text{если } I\geqslant s,
\end{array}
\right.
\end{equation*}
где $I$, $S$, $s$ --- это соответственно уровень запасов в начале
месяца, после поступления заказа и критический.

При возникновении спроса на товар он немедленно удовлетворяется,
если уровень запасов, по меньшей мере, равен спросу на товар. Если
спрос превышает уровень запасов, поставка той части товара,
которая превышает спрос над предложением, откладывается и
выполняется при будущих поставках. (В этом случае новый уровень
запасов равен старому уровню запасов минус объем спроса, что
приводит к появлению отрицательного уровня запасов.) При
поступлении заказа товар в первую очередь используется для
максимально возможного выполнения отложенных поставок (если
таковые имеются); остаток заказа (если таковой имеется)
добавляется в запасы.

В большинстве систем управления запасами, наряду с прямыми
затратами по приобретению товаров, возникают дополнительные
затраты --- затраты на хранение, а также издержки, связанные с
нехваткой товара.

Пусть $I(t)$ --- уровень запасов в момент времени $t$ (величина
$I(t)$ может быть положительной, отрицательной или равняться
нулю); $I^{+}(t)=\max\{I(t),0\}$ --- количество товара, имеющегося
в наличии в системе запасов на момент времени $t$,
$I^{+}(t)\geqslant 0$, а $I^{-}(t)=\max\{-I(t),0\}$ --- количество
товара, поставка которого была отложена на момент времени $t$
$(I^{-}(t)\geqslant 0.)$

Предполагается, что затраты $h$ на хранение в месяц составляют 1
д.е. на единицу товара, имеющегося в (положительных) запасах.
Затраты на хранение включают арендную плату за склад, страховки,
расходы на обслуживание и налоги, а также скрытые издержки,
возникающие, если капитал, вложенный в запасы, мог бы
инвестироваться куда-нибудь еще. Если $I^{+}(t)$ представляет
количество товара в запасах на момент времени $t$, то среднее по
времени количество товара, находящегося в запасах в течение $n$
месяцев, составляет
\begin{equation*}
\bar{I}^+=\dfrac{\int\limits_0^nI^+(t)dt}{n}.
\end{equation*}
Тогда средние затраты на хранение в месяц составляют $h\bar{I}^+$
д.е.

Допустим теперь, что издержки $\pi$, связанные с отложенными
поставками, равны 5 д.е. на единицу товара в отложенной поставке
за месяц. Тогда среднее по времени количество товара в отложенных
поставках составит
\begin{equation*}
\bar{I}^-=\dfrac{\int\limits_0^nI^-(t)dt}{n}.
\end{equation*}
Средние издержки,образовавшиеся в связи с отложенными поставками,
в месяц будут составлять $\pi \bar{I}^-$.

Предположим, что исходный уровень запасов $I(0)=60$ и у компании
нет неприобретенных заказов. Смоделировать работу системы в
течение $n=120$ мес. и воспользоваться показателями средних общих
расходов в месяц (которые включают в себя сумму средних затрат на
приобретение заказа в месяц, средних затрат на хранение в месяц и
средних издержек, связанных с нехваткой товара, в месяц), чтобы
сравнить следующие девять стратегий осуществления заказов ($s$ ---
точка заказа): \\
\begin{center}
\begin{tabular}{c|ccccccccc}
\hline s & 20 & 20 & 20 & 20 & 40 & 40 & 40 & 60 & 60 \\
\hline S & 40 & 60 & 80 & 100 & 60 & 80 & 100 & 80 & 100 \\
\hline
\end{tabular}
\end{center}

\newpage
\section{Моделирование производственной системы}
 Компания собирается построить новое предприятие, состоящее из
станции приема-отправки и пяти рабочих станций (РИСУНОК).
\begin{figure}[!h]
\epsfig{file=sxema_stankov.eps, height=9.92 cm, width=16.94 cm}
\caption{Схема производственной системы}
\end{figure}

На одной станции все станки одинаковы, но станки, установленные на
разных станциях, отличаются. Расстояние между шестью станциями
приведено в таблице.
\begin{table}[!h]
\centering
\let\PBS=\PreserveBackslash
\renewcommand{\multirowsetup}{\centering}
\setlongtables \vspace{-1mm}
\begin{tabular}{|>{\PBS \center} m{2 cm}|>{\PBS \center}m{1 cm}|>{\PBS \center}m{1 cm}|>{\PBS \center }m{1 cm}|>{\PBS \center}m{1 cm}|>{\PBS \center}m{1 cm}|>{\PBS \center}m{1 cm}|}
%\multicolumn{2}{r}{}\\
\multicolumn{7}{c}{Таблица 1 --- Расстояние между шестью станциями, метры} \\
\hline
Станция  &  1&2&3&4&5&6 \\
\hline
1 & 0 & 150 & 213 & 336 & 300 & 150 \\
2 & 150 & 0 & 150 & 300 & 336 & 213 \\
3 & 213 & 150 & 0 & 150 & 213 & 150 \\
4 & 336 & 300 & 150 & 0 & 150 & 213 \\
5 & 300 & 336 & 213 & 150 & 0 & 150 \\
6 & 150 & 213 & 150 & 213 & 150 & 0\\
\hline
\end{tabular}
\end{table}
Заготовки поступают на станцию приема-отправки с интервалами между
прибытиями, представленными независимыми экспоненциально
распределенными случайными величинами со средним значением 1/15 ч.
Существует три типа работ для заготовок; работы относятся к типу
1, 2 и 3 с вероятностью 0.3, 0.5 и 0.2, соответственно. Для работ
типа 1, 2 и 3 требуется соответственно выполнение четырех, трех и
пяти операций, все операции должны производиться на указанных
рабочих станциях в установленном порядке. Каждая заготовка
изначально попадает на станцию приема-отправки, откуда она
начинает свой маршрут в системе, по завершении которого она
покидает систему через станцию приема-отправки. Маршруты заготовок
для различных типов работ указаны в ТАБЛИЦЕ. \\
\begin{table}[!h]
\centering
\let\PBS=\PreserveBackslash
\renewcommand{\multirowsetup}{\centering}
\setlongtables \vspace{-1mm}
\begin{tabular}{|>{\PBS \flushleft}m{2 cm}|>{\PBS \flushleft }m{7 cm}|}
%\multicolumn{2}{r}{}\\
\multicolumn{2}{c}{Таблица 2 --- Маршруты заготовок для трех типов работ} \\
\hline
Тип работы & Рабочие станции на маршруте заготовки \\
\hline
1 & 3, 1, 2, 5 \\
2 & 4, 1, 3\\
3 & 2, 5, 1, 4, 3 \\
\hline
\end{tabular}
\end{table}
Заготовка будет перемещаться с одной станции на другую при помощи
автопогрузчика, который двигается с постоянной скоростью
$1.52~{\text{м/с}}$. Когда автопогрузчик освобождается, он
обслуживает заготовки в порядке увеличения расстояния между
автопогрузчиком и заготовкой, подлежащей перевозке (то есть
ближайшие к нему заготовки перемещаются в первую очередь). Если в
тот момент, когда требуется переместить заготовку, свободны
несколько автопогрузчиков, используется ближайший к ней
автопогрузчик. Когда автопогрузчик завершает перемещение заготовки
на рабочую станцию, он остается на этой станции, если в системе
нет невыполненных запросов для заготовок. Если при поступлении
заготовки на станцию все станки на ней уже заняты или
заблокированы, заготовка помещается в единственную очередь с
дисциплиной обслуживания FIFO на этой станции. Время выполнения
операции определенным станком представлено переменной, которая
имеет гамма-распределение с параметром формы 2, ее среднее
значение зависит от того, какой тип работы выполняется, и от того,
на какой рабочей станции установлен станок. Среднее время
обслуживания для каждой заготовки и каждой операции дано в
таблице.
\begin{table}[!h]
\centering
\let\PBS=\PreserveBackslash
\renewcommand{\multirowsetup}{\centering}
\setlongtables \vspace{-1mm}
\begin{tabular}{|>{\PBS \flushleft}m{2 cm}|>{\PBS \flushleft }m{11 cm}|}
%\multicolumn{2}{r}{}\\
\multicolumn{2}{c}{Таблица 3 --- Среднее время обслуживания для каждой заготовки и каждой операции} \\
\hline
Тип работы & Среднее время обслуживания в ходе последовательных операций, ч  \\
\hline
1 & 0.25, 0.15, 0.10, 0.30  \\
2 & 0.15, 0.20, 0.30 \\
3 & 0.15, 0.10, 0.35, 0.20, 0.20 \\
\hline
\end{tabular}
\end{table}
Таким образом, среднее совокупное время обслуживания, определенное
для всех заготовок, равно 0.77~ч. Когда станок завершает обработку
заготовки, она блокирует этот станок (то есть станок не может
приступить к обработке другой заготовки) до тех пор, пока ее не
заберет автопогрузчик.

Требуется \\
1) смоделировать предложенную производственную систему, чтобы
определить, {\bf а)} сколько станков необходимо установить на
каждой рабочей станции и {\bf б)} какое количество автопогрузчиков
требуется, чтобы достичь ожидаемой производительности 120
заготовок за 8-часовой рабочий день; \\
2) среди тех проектов системы, которые позволяют достичь нужной
производительности, выбрать наилучший, исходя из таких показателей
работы, как среднее время пребывания заготовки в системе,
максимальные размеры очередей на входе, период времени, в течение
которого рабочая станция занята.


Рассмотрим упрощенную версию задачи с планированием конфигурации производственной системы:

Заготовки поступают на станцию приема-отправки с интервалами 
между прибытиями, представленными независимыми экспоненциально  
распределенными случайными величинами со средним значением 1/15 ч. Существует три типа работ для  
заготовок; работы относятся к типу 1, 2 и 3 с вероятностью соответственно 0.3; 0.5 и 0.2. 
Для работ типа 1,2 и 3 требуется выполнение следующих операций на указанных рабочих станциях:
 
\begin{table}[!h]
	\centering
	\let\PBS=\PreserveBackslash
	\renewcommand{\multirowsetup}{\centering}
	\setlongtables \vspace{-1mm}
	\begin{tabular}{|>{\PBS \flushleft}m{2 cm}|>{\PBS \flushleft }m{4 cm}|}
		%\multicolumn{2}{r}{}\\
		\multicolumn{2}{c}{Таблица --- Маршруты заготовок для трех типов работ} \\
		\hline
		Тип работы & Рабочие станции на маршруте заготовки \\
		\hline
		1 & 1 \\
		2 & 3\\
		3 & 1\\
		\hline
	\end{tabular}
\end{table}

Если при поступлении заготовки на станцию все станки на ней уже заняты или заблокированы, заготовка помещается в единственную очередь с дисциплиной обслуживания FIFO на этой станции. Время выполнения операции определенным станком представлено переменной, которая имеет гамма-распределение с  
параметром формы 2, ее среднее значение зависит от того, какой тип работы выполняется, и от того, на какой рабочей станции установлен станок. Среднее время  
обслуживания для каждой заготовки и каждой операции дано в таблице
\begin{table}[!h]
	\centering
	\let\PBS=\PreserveBackslash
	\renewcommand{\multirowsetup}{\centering}
	\setlongtables \vspace{-1mm}
	\begin{tabular}{|>{\PBS \flushleft}m{2 cm}|>{\PBS \flushleft }m{11 cm}|}
		%\multicolumn{2}{r}{}\\
		\multicolumn{2}{c}{Таблица --- Среднее время обслуживания для каждой заготовки и каждой операции} \\
		\hline
		Тип работы & Среднее время обслуживания в ходе последовательных операций, ч  \\
		\hline
		1 & 0.25\\
		2 & 0.15\\
		3 & 0.10\\
		\hline
	\end{tabular}
\end{table}

Когда станок завершает обработку заготовки, она уходит с рабочей станции. 

Нам нужно смоделировать предложенную производственную систему, чтобы 
оценить количество заготовок каждого типа за 8-часовой рабочий день.

\begin{center}
	Методические указания к задаче 
\end{center}
1. Генерируем время появления детали и определяем тип детали.
2. Распределяем деталь в очередь к станку в зависимости от ее номера. Генерируем время обработки детали -- ухода.
2.1 Генерирование очереди
